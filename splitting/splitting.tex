\documentclass[12pt,magyar,a4paper]{article}
\usepackage[utf8]{inputenc}
\usepackage{graphicx}
\usepackage{listings}
\usepackage{amsmath}
\usepackage{amssymb}
\usepackage{ae,aecompl}
\usepackage{fix-cm}

\newcommand{\leftexp}[2]{{\vphantom{#2}}^{#1}{#2}}

\title{Splitting keresése D-szimbólumokban}
\author{Boróczki, Lajos}
\date{2013. január 29.}

\begin{document}
\maketitle
\tableofcontents

\section{Áttekintés}
3 dimenziós baricentrikus szimplexekkel történő kövezések esetén korábban már
megvizsgáltuk, hogy milyen 2 dimenziós kövezést indukál a D-szimbólum egy-egy
szimplex-csúcs osztály egy kis környezetében és ezek mit jelentenek az eredeti 3
dimenziós kövezésre nézve (ez volt a rész D-szimbólum eset, ahol csak a
szférikus, illetve 0,3 esetben az euklideszi kövezést tartottuk meg, mint
érdekes esetet.) A vizsgálatot eredetileg a rész D-szimbólumok irányából
indítottuk, de a gondolat a másik irányból is adja magát.

A kérdés a következőképpen általánosítható: Milyen 2 dimenziós kövezést indukál
az összefüggő D-szimbólum a szimplex-csúcs osztályok egy olyan 2-partícióján,
ahol mindkét partíció összefüggő? (2-partíció: két összefüggő szimplex-csúcs
osztály halmaz, melyek metszete üres és uniója a teljes szimplex-csúcs osztály
halmaz.)

Splittinget ($S^2$ vagy $\mathbb{E}^2$) akkor találunk, ha a 2-partíció egyik partíciója sem
1 elemű és nem hiperbolikus kövezést találunk rajta. Fontos megjegyezeni, hogy
előfordulhat olyan eset is, hogy a talált $\mathbb{E}^2$ kövezés nem egy végtelen távoli
pontot jelez, hanem egy henger/tórusz/fibrum jellegű alakzatot, ezzel a
kérdéssel egyelőre csak észrevétel szinten foglalkoztam. Ezenkívül
előfordulhatnak $S^2$ esetben rossz orbifoldok.


\section{A lehetséges 2-partíciók felsorolása}
Ez egy "egyszerű" gráfelméleti feladat, ahol a csúcsok a szimplex-csúcs
osztályok, az élek a szimplex-él osztályok (itt nem lényeges, hogy milyen
színűek az élek): Ha tudjuk, hogy a gráf összefüggő, akkor annyi a feladatunk,
hogy felsoroljuk az összes összefüggő részgráfot, amiben egy adott csúcs
szerepel (ez a kiindulási csúcs.) Majd ellenőrizzük, hogy az összefüggő részgráf
komplementere összefüggő-e, ha igen, találtunk egy 2-partíciót. Az előbbi összes
összefüggő részgráfot egy mélységi bejárással meg lehet találni, míg az utóbbi
összefüggőséget egy szélességi bejárással lehet ellenőrizni. Az algoritmus
a csúcsok száma szerint $n^3$ időben fut (a mélységi bejárás $n$, a szélességi
átlagosan $(n/2)^2$).


\section{Adott 2-partíción indukált kövezés}
A két partíció között kialakuló 2 dimenziós kövezésre szeretnénk felírni a
kombinatorikus görbület függvényt, amivel egyszerűen meg tudjuk állapítani,
milyen kövezéssel állunk szemben. Ehhez szükségünk van a generátorokra és
relációkra.

Vezessük be az eredeti térkitöltéshez (vagy D-szimbólumhoz) és az adott
2-partícióhoz értelmezett "elvágott szimplex-él osztályok" fogalmát: Azokat a
szimplex-él osztályokat nevezzük így, melyek egyik vége az egyik, másik vége a
másik partícióban szerepel. (Pontosan fogalmazva egyik illetve másik "hozzá
tartozó szimplex-csúcs osztályt" kéne említeni.)

Az indukált 2 dimenziós kövezésben az elvágott szimplex-él osztályok fogják a
csúcsokat jelenteni, az eredeti kövezésben körülöttük lévő forgatás vagy
tükrözés jelenti az indukált kövezés generátorait, végül az eredeti kövezésben
ezek relációi jelentik az indukált kövezésben a relációkat.

Az eredeti kövezésben a szimplex-lap osztályoknak két fajtája van, vagy nincs
benne elvágott szimplex-él osztály, vagy pontosan 2 van benne. Az előbbi az
indukált kövezés szempontjából nem érdekes, mert teljes egészében 1 partícióban
van, az utóbbiak számunkra érdekesek, mert az indukált kövezés éleit fogják
adni.

Hasonlóan megvizsgálva az eredeti kövezés szimplex osztályait, ezeknek 3 fajtája
van: vagy nincs benne elvágott szimplex-él osztály, vagy 3 elvágott szimplex-él
osztály van, vagy 4 elvágott szimplex-él osztály van. Az elvágott szimplex-él
osztály nélküliek nem fontosak, a 3 elvágott szimplex-él osztállyal rendelkezők
háromszögeket adnak az indukált kövezésben, végül a 4 elvágott szimplex-él
osztállyal rendelkezők négyszögeket adnak az indukált kövezésben.

A kombinatorikus görbület függvényt kicsit módosítanunk kell, ugyanis az
háromszögelésre vonatkozik; de szerencsére a négyszögeket tudjuk egyszerűen
háromszögelni, ez esetben a képletben a négyszög komponensekre a
$-2+1/m_{01}+1/m_{12}+1/m_{23}+1/m_{30}$ képletet lehet alkalmazni a
$-1+1/m_{01}+1/m_{12}+1/m_{20}$ helyett:

\begin{align*}
  K=\sum_{\mathrm{triangles}}-1+\frac{1}{m_{01}}+\frac{1}{m_{12}}+\frac{1}{m_{20}} + 
  \sum_{\mathrm{quadrangles}}-2+\frac{1}{m_{01}}+\frac{1}{m_{12}}+\frac{1}{m_{23}}+\frac{1}{m_{30}}
  \begin{array}{cccc}
    > & & S^2 \\
    = & 0 & \mathbb{E}^2 \\
    < & & H^2
  \end{array}
\end{align*}
\end{document}
