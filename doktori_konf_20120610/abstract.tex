\documentclass[a4paper]{article}
\usepackage[english]{babel}
\usepackage[T1]{fontenc}
\usepackage[utf8]{inputenc}
\begin{document}
\small
TÁMOP Doktorandusz konferencia\\
Budapest/Hungary, June 11-13, 2012
\vfill
\large
\begin{center} 
  Face and edge transitive tilings in three dimensional non-Euclidean spaces
\end{center}
\vfill
\small
\textbf{BORÓCZKI, Lajos}\\
Budapest University of Technology and Economics, MI Department of Geometry\\
E-mail: boroczki.lajos@gmail.com\\
http://www.math.bme.hu/~geometry
\vfill
\small
\textbf{Abstract.} Any tiling, generated by a crystallographic group with
compact fundamental domain, can be represented by a diagram and a matrix valued
function, based on their barycentric subdivision and the adjacency relations
between the orbits and the particular simplices. The representation is called
the D-symbol of a tiling in honour of Delone, Delaney and Dress. The
representation is easily adoptable to computer programs.

Both face and edge transitive tilings are special tilings, which have only 1
face or edge orbit respectively in the crystallographic group. (Our fundamental
domain doesn't have to be the smallest possible, it can have inner symmetries.)
It's easy to prove that there are finitely many possible diagrams in these
cases, so enumerating all of them in every possible geometry seems possible.

The problem in the non-Euclidean planes (E. Molnár with  Z.  Lu\u{c}i\'{c} and
M. Stojanovi\'{c} (1994)) and in the space $E^3$ (E. Molnár with A.W.M. Dress
and D.H. Huson (1993)) is already solved.

We can enumerate tilings with at most 18 simplex-orbits (out of 24). Three
dimensional tilings induce some two dimensional tilings around the vertices of
the simpleces and between any two partition of the simplex-vertices. These can
are examined by a simple formula, which gives us finitely many possible matrix
valued functions for every diagram to investigate.

Based on the Thurston-theorem there are 8 possible geometries.  There exists at
least 4 proof of the theorem but none of them is constructive. Based on our
method it would be easier to find a tiling which does not fit in any of the 8
geometries; but inspecting the tilings we can possibly move forward to a
constructive proof of the theorem.

But there are some more problems to solve first: we have to find the possible
splittings in a D-symbol (which corresponds to tori in Thurston's notation)
and we have to be able to tell the signature of the projective space of the
tiling of a primitive D-symbol.

The results descussed above are supported by the grant TÁMOP -
4.2.2.B-10/1--2010-0009.
\vfill
\vfill
\end{document}
