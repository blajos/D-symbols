\usepackage[english]{babel}
\usepackage[T1]{fontenc}
\usepackage[utf8]{inputenc}
\usepackage{graphicx}
\usepackage{listings}
\usepackage{amsmath}
\usepackage{amssymb}
%\usepackage{amsthm}
\usepackage{ae,aecompl}
\usepackage{fix-cm}

\usetheme{Goettingen}
%\usetheme{Singapore}

\beamertemplatenavigationsymbolsempty

\title{Face and edge transitive tilings in three dimensional non-Euclidean
spaces}
\author{Boróczki, Lajos}
\date{Budapest, 11-13 June 2012 \\{\tiny Workshop of the BME Graduate School of
Mathematics and Computer Science}}

\begin{document}

\begin{frame}
  \maketitle
\end{frame}

\begin{frame}
  \mode<presentation>{\frametitle{Outline}}
  \tableofcontents
\end{frame}
\newpage

\section{Abstract}
\begin{frame}
  \tiny
Workshop of the BME Graduate School of
Mathematics and Computer Science\\
Budapest, 11-13 June 2012
\vfill
\normalsize
\begin{center} 
  Face and edge transitive tilings in three dimensional non-Euclidean spaces
\end{center}

\vfill
\tiny
\textbf{BORÓCZKI, Lajos}\\
Budapest University of Technology and Economics, MI Department of Geometry\\
E-mail: boroczki.lajos@gmail.com\\
\vfill

http://www.math.bme.hu/~boroczki
\vfill

\tiny
\textbf{Abstract.}  Any tiling, generated by a crystallographic group with
compact fundamental domain, can be represented by a diagram and a matrix valued
function, based on their barycentric subdivision and the adjacency relations
between the orbits and the particular simplices. The representation is called
the D-symbol of a tiling in honour of Delone, Delaney and Dress. The
representation is easily adaptable to computer programs.
\vfill

Both face and edge transitive tilings are special tilings, which have only 1
face or edge orbit respectively in the crystallographic group. (Our fundamental
domain doesn't have to be the smallest possible, it can have inner symmetries.)
It's easy to prove that there are finitely many possible diagrams in these
cases, so enumerating all of them in every possible geometry seems possible.
\vfill

The problem in the non-Euclidean planes (E. Molnár with  Z.  Lu\u{c}i\'{c} and
M. Stojanovi\'{c} (1994)) and in the space $E^3$ (E. Molnár with A.W.M. Dress
and D.H. Huson (1993)) is already solved.
\vfill

We can enumerate tilings with at most 18 simplex-orbits (out of 24). Three
dimensional tilings induce some two dimensional tilings around the vertices of
the simplices and between any two partition of the simplex-vertices. These can
be examined by a simple formula, which gives us finitely many possible matrix
valued functions for every diagram to investigate.
\vfill

Based on the Thurston-theorem there are 8 possible geometries.  There exists at
least 4 proof of the theorem but none of them is constructive. Based on our
method it would be easier to find a tiling which does not fit in any of the 8
geometries; but inspecting the tilings we can possibly move forward to a
constructive proof of the theorem.
\vfill

But there are some more problems to solve first: we have to find the possible
splittings in a D-symbol (which corresponds to tori in Thurston's notation)
and we have to be able to tell the signature of the projective space of the
tiling of a primitive D-symbol.
\vfill

The results discussed above are supported by the grant TÁMOP -
4.2.2.B-10/1--2010-0009.

\end{frame}

Egy kis vazlat:
innen max. 7 slide-om lehet, de inkabb 6
1. D-symbol bemutatasa I.
2. Face/Edge-transitive D-symbol bemutatasa II.
3. D-symbol illusztraciok
4. Kisebb dimenzios resz D-symbolumok vizsgalata
5. Ellenkezo irany, eredmenyek
6. Thurston-sejtes/tetel, celok
7. Tovabbi megoldando problemak
8. Questions?

\section{D-symbols}
\begin{frame}
  D-symbols:
  \begin{itemize}
    \item Based on the baricentric subdivision of a tiling
    \item Structure:
      \begin{itemize}
	\item D-diagram: $dim+1$ colored graph, which represents adjacencies of
	  simplex-orbits
	\item Matrix function on simplex orbits, which represents the number of
	  simpleces (not orbits) around a $dim-2$ dimensional edge.
      \end{itemize}
    \item Constraints:
      \begin{itemize}
	\item Compatibility between the diagram and the matrix function.
	\item Compatibility with baricentric subdivision
      \end{itemize}
  \end{itemize}
\end{frame}

\section{Finding D-diagrams}
\begin{frame}
  Simple algorithm for finding D-diagrams:
  \begin{itemize}
    \item Fixed dimension and cardinality
    \item Take a fixed number of graph vertices and $dim+1$ number of colors
    \item Enumerate every possible edge-combinations and drop the "bad" and
      duplicate combinations.
    \item The algorithm can find the possible edge-transitive diagrams with a
      little perturbation.
    \item Constraints:
      \begin{itemize}
	\item Rank of vertices
	\item Patterns which makes the diagram incompatible with baricentric
	  subdivision
	\item Is the diagram connected?
	\item Permutation of vertices (ordering of diagrams)
      \end{itemize}
  \end{itemize}
\end{frame}

\section{Finding matrix-functions}
\begin{frame}
  Algorithm for finding possible matrix-functions of a D-diagram in
  $3$-dimensions:
  \begin{itemize}
    \item Parametric matrix function, minimal values
    \item $2$-dimensional subsymbols of 3 colors (simplex point stabilizers):
      \begin{itemize}
	\item Based on combinatorial curvature (decreases if we increase the
	  parameter values)
	  \begin{align*}
	    K(\leftexp{c}{\mathcal{D}}^i)=\sum_{D\in
	    \leftexp{c}{\mathcal{D}}^i}\left(-1+\sum_{\substack{0\le j<k\le 3 \\
	    j,k\ne i}}\frac{1}{m_{jk}(D)}\right)
	    \begin{array}{cccc}
	      > & & S^2 \\
	      = & 0 & \mathbb{E}^2 \\
	      < & & H^2
	    \end{array}
	  \end{align*}
	\item Only spherical or euclidean
	\item If spherical we have to exclude bad orbifolds
      \end{itemize}
    \item Increase every parameter one by one and manage infinite series
  \end{itemize}
  Some easy to answer questions:
  \begin{itemize}
    \item Does the D-symbol have proper inner symmetry (diagram and matrix function)?
    \item Does the tiling have ideal point for a given D-symbol?
  \end{itemize}
\end{frame}
\end{document}
