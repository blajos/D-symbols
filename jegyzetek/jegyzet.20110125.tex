\documentclass[12pt,magyar,a4paper]{article}
\usepackage{graphicx}
\usepackage[magyar]{babel}
\usepackage[T1]{fontenc}
\usepackage[utf8]{inputenc}
\usepackage{graphicx}
\usepackage{listings}
\usepackage{amsmath}
\usepackage{amssymb}
%\usepackage{amsthm}
\usepackage{babel}
\usepackage{ae,aecompl}

%\theoremstyle{plain}
\newtheorem{thm}{Tétel}[section]
\newtheorem{thmsub}{Tétel}[subsection]
\newtheorem{lem}[thm]{Lemma}
\newtheorem{lemsub}[thmsub]{Lemma}
\newtheorem{all}[thm]{Állítás}
\newtheorem{allsub}[thmsub]{Állítás}
\newtheorem{kov}[thm]{Következmény}
\newtheorem{sej}[thm]{Sejtés}

%\theoremstyle{definition}
\newtheorem{defn}[thm]{Definíció}
\newtheorem{defnsub}[thmsub]{Definíció}
\newtheorem{jel}[thm]{Jelölés}

%\theoremstyle{remark}
\newtheorem{alg}[thm]{Algoritmus}
\newtheorem{megj}[thm]{Megjegyzés}
\newenvironment{biz}{Bizonyítás: }{$\square$}

\begin{document}

\title{Jegyzet 2011.01.25.}
\author{Boróczki Lajos}

\maketitle

\section{Diplomamunkából újra felhasználandó anyagok}
Egyelőre csak címszavakban.

\subsection{Bevezetés}
Tetszőleges kövezésből készíthető algebrai leírás baricentrikus felbontás
segítségével. A témánk a lehetséges algebrai leírások alapján visszakapható
kövezésekről szól.

\subsection{D-szimbólum rövid ismertetése és felépítése egy példa alapján}
\subsubsection{D-szimbólum felépítése}
Baricentrikus felbontás.

Szomszédsági operációk.

$\mathcal{R}$ és $\mathcal{M}$ mátrixfüggvények.

\subsubsection{Példa}
Az $\mathbb{E}^3$ tér négyzetes hasábbal történő egyik legegyszerűbb
térkitöltése.

\subsection{D-szimbólum követelmények}
\label{D-sym-cond}
Az összes 3 dimenziós geometriában az összes lehetséges kövezés felsorolásához
szeretnénk a D-szimbólumokat felhasználni. A következő szükséges feltételeket
támasztjuk a D-szimbólumokkal szemben:
\begin{itemize}
  \item D-szimbólum absztrakt definíciója
  \item Megkötések $\mathcal{R}$ és $\mathcal{M}$ mátrixfüggvényekre (valódi
    vagy végtelen távoli szimplex csúcsok, jó orbifold feltételek)
\end{itemize}

\subsection{D-szimbólumok felsorolásához szükséges algoritmusok}
\subsubsection{D-diagramok felsorolása}
Rendezés bevezetése a D-diagramokra.

A használt algoritmus ismertetése. (Döntési fa modellel egyszerűen
szemléltethető.)

\subsubsection{A D-diagramokhoz tartozó mátrixfüggvények felsorolása}
Az $\mathcal{R}$ mátrixfüggvény és a paraméterek felírhatóak a D-diagram
alapján. A lehetséges $\mathcal{M}$ mátrixfüggvények illetve a hozzájuk tartozó
paraméterértékek pedig véges sokan vannak vagy végtelen láncokat alkotnak (lásd
\ref{D-sym-cond} fejezet: megkötések a mátrixfüggvényekre).

Rendezés bevezetése a teljes D-szimbólumra.

Algoritmus ismertetése.

\section{Továbblépések}
Ezek csak jegyzetek, gondolatmenetek.

\subsection{Adott D-szimbólumhoz tartozó fundamentális tartomány előállítása}
FIXME Megtehető konvex módon. Kérdés, hogy a ragasztás során végig konvex
marad-e a rendszer, vagy csak a legvégén. (Diplomamunkabeli D.109 esetén pl.
nem tud a rendszer végig konvex maradni.)

Mikor tud elromlani a konvexitás (ezeket a helyzeteket ki kell küszöbölni):
\begin{itemize}
  \item Pontosan akkor lesz az alakzat konkáv, ha van konkáv lapszöge
  \item Ha adott egy él, amit szimplexekkel körbe tudunk rakni ismétlés nélkül,
    a szimplexek "felénél többet", de nem az összeset ragasztjuk körbe az él
    körül. Ilyen éleket az irányitott (tükrözés mentes) 1 értékű paraméterek
    jelentenek.
  \item Ha egy élet csak ismétléssel tudunk körbe rakni, akkor nem tud konkáv
    lenni a hozzá tartozó lapszög.
\end{itemize}

További gondolatok:
\begin{itemize}
  \item Tükörképeket nem ragasztunk (ugyanazt a szimplexet 1-szer hasznaljuk fel
    a fundamentális tartomány ragasztásakor)
  \item Ha egy él körülrakásában van tükrözés, nem okoz gondot az él, mert lesz
    ismétlés a körberakásában.
  \item Ha nincs tükrözés, de legálabb 2-szer körbemegyünk (azaz a paraméter
    értéke legalább 2), szintén nem okoz gondot az él, mert lesz ismétlés a
    körberakásában.
  \item Ha 1 élet teljesen körbe rakunk szimplexekkel, akkor a fundamentális
    tartomány egy "belső élévé" változik, így nem alakít ki lapszöget.
  \item Azokat az eseteket kell tehát csak megvizsgálni, ahol több élet is körbe
    lehet rakni szimplexekkel ismétlés nélkül (vagyis irányított 1 paraméterű
    él-körből több is van.)
\end{itemize}

Az utolsó további gondolat folytatása és egy leendő algoritmus:
\begin{itemize}
  \item Vesszük az irányított 1 értékű paramétereket (él-körök), az általuk
    érintett szimplexek száma szerint csökkenő sorrendben
  \item Az első ilyenhez tartozó élet körbe ragasztjuk szimplexekkel.
  \item Megnézzük, hogy a maradék él-körök közül hány olyan van, aminek csak 1
    közös szimplexe van az eddig kiválasztott körök uniojával, vesszük a
    legnagyobb ilyet és kezdjük elölről az algoritmust (lásd \ref{ftrajz1}. ábra.)
    \begin{figure}
      \caption{\label{ftrajz1} Két irányított kör, 1 közös szimplexszel.}
      \center
      \includegraphics[width=0.4\textwidth]{fund-tart_rajz1.pdf}
    \end{figure}
  \item Megnézzük, hogy a maradék él-körök közül hány olyan van, aminek legalább
    2 közös szimplexe van az eddig kiválasztott körök uniojával. Vesszük azt,
    aminek a legtöbb szimplexe van még szabadon, és kétoldalról elkezdjük
    összerakni a kört (lásd \ref{ftrajz2}. ábra.) Fontos, hogy a maradék
    kört nem tudjuk körbe összeragasztani mert csak ekvivalens éleket jár
    körbe a két útvonal, de nem ugyanazt.
    \begin{figure}
      \caption{\label{ftrajz2} Két irányított kör, 2 közös szimplexszel.}
      \center
      \includegraphics[width=0.4\textwidth]{fund-tart_rajz2.pdf}
    \end{figure}
  \item A maradék szimplexek között megnézzük, hogy van-e még él-kör, ha igen,
    a maradékra is végrehajtjuk a fenti algoritmust
  \item Mostanra kaptunk néhány (esetleg 1 elemű) komponenst, melyeket szabadon
    összeragaszthatunk egy-egy operáció mentén, mert a "veszélyes" éleket már
    hatástalanítottuk.
\end{itemize}

\subsection{Generátorok és relációk}
Először a fundamentális tartományt kell felépíteni.

\subsection{Schönfliess-Bieberbach tétel általánosítása}
Sch-B tétel kimondása.

A Schönflies-Bieberbach tételt általánosíthatjuk tetszőleges geometriára
úgy, hogy rácsok helyett fixpont mentes egybevágóságokat vizsgálunk.

Írunk egy módszert, ami egy adott D-szimbólumhoz előállít egy fixpont mentes
egybevágóságcsoportot leíró "felfújt" D-szimbólumot.

Az algoritmus:
\begin{itemize}
  \item Sorra vesszük a tükrözéseket (lexikografikusan első az operáció száma,
    második a szimplex sorszáma,) és minden tükrözésre megduplázzuk a
    D-szimbólumot. Az eredeti és az új D-szimbólumot az adott tükrözéshez
    tartozó operáció köti csak össze. (Ilyenkor el tud romlani a nem szomszédos
    operációk által meghatározott élek körberakásának feltétele... lásd
    \ref{schrajz1}. ábra.) Ez nem elég.
    \begin{figure}
      \caption{\label{schrajz1} Tükrözések felbontása 1.}
      \center
      \includegraphics[width=0.3\textwidth]{sch-b-alt_rajz1.pdf}
    \end{figure}
  \item Javított első lépés: Megnézzük az ekvivalens tükrözéseket. Mit is jelent
    ez: A súlyzó alakú rendszereket ki kell javítani vele (lásd \ref{schrajz2}.
    ábra.) Hogyan lehet továbblépni?
    \begin{figure}
      \caption{\label{schrajz2} Tükrözések felbontása 2.}
      \center
      \includegraphics[width=0.3\textwidth]{sch-b-alt_rajz2.pdf}
    \end{figure}
  \item Új verzió: Az összes élet vizsgáljuk:
    \begin{enumerate}
      \item Sorra vesszük a nem iranyítottan (tükrözés segítségével) körberakott
	éleket. Minden ilyet megszüntetünk úgy, hogy a körberakáshoz tartozó
	paraméter duplájaszor lemásoljuk a teljes D-szimbólumot, és a
	tükrözött szimplexeket és másolataikat kapcsoljuk össze a tükrözés
	operátorával (lásd \ref{schrajz3}. ábra.)
	\begin{figure}
	  \caption{\label{schrajz3} Irányítatlan kör felbontása}
	  \center
	  \includegraphics[width=0.75\textwidth]{sch-b-alt_rajz3.pdf}
	\end{figure}
      \item Sorra vesszük az irányítottan (tükrözés nélkül) körberakott éleket.
	Minden ilyet megszüntetünk úgy, hogy először szétvágjuk a kört egy
	tetszőleges helyen, majd a körberakáshoz tartozo paraméterszer
	lemásoljuk a teljes szétvágott D-szimbólumot, és a vágáshoz betoldjuk
	(lásd \ref{schrajz4}. ábra.)
	\begin{figure}
	  \caption{\label{schrajz4} Irányított kör felbontása}
	  \center
	  \includegraphics[width=0.75\textwidth]{sch-b-alt_rajz4.pdf}
	\end{figure}
    \end{enumerate}
\end{itemize}


\end{document}
