\documentclass[12pt,magyar,a4paper]{article}
\usepackage{graphicx}
\usepackage[magyar]{babel}
\usepackage[T1]{fontenc}
\usepackage[utf8]{inputenc}
\usepackage{verbatim}
\usepackage[colorlinks=true,linkcolor=black,urlcolor=blue]{hyperref}
\usepackage{listings}
\usepackage{amsmath}
\usepackage{amssymb}
\usepackage{ae,aecompl}

%\hyphenpenalty=5000
%\tolerance=1000

\begin{document}

\title{FIXME D-szimbólumok algoritmikus vizsgálata}
\author{Boróczki, Lajos}
\date{2011 február}
\maketitle

\begin{abstract}
  A térkitöltések kezelésének egy algebrai módszerével, a Delone-Delaney-Dress
  szimbólumokkal (röviden D-szimbólum) foglalkozunk. A módszer (\cite{xy}
  alapján) a térkitöltéseket egy több színű gráffal és a hozzá tartozó
  mátrix-függvényekkel írja le.

  Alapvető célunk a 3 dimenziós D-szimbólumok algoritmikus felsorolása, illetve
  a hozzájuk tartozó lehetséges kövezések minél részletesebb jellemzése.
  Végtelen sok D-szimbólum van rögzített dimenzióban is, ezért bemutatunk egy
  olyan megszorítási lehetőséget is, ami garantálja a véges sok esetet, de
  továbbra is érdekes. Általános esetben jelenleg legfeljebb 12, éltranzitív
  esetben a 24-ből legfeljebb 18 elemű D-szimbólumokat tudjuk felsorolni.

  A D-szimbólumok által leírt kövezések pontos geometriai jellemzőit egyelőre
  nem tudjuk algoritmikusan leírni, ugyanis ez a Thurston-sejtés témaköréhez
  vezet, aminek nem ismert konstruktív bizonyítása (\cite{xy}). Ezért
  néhány esetben a kézzel történő jellemzést megmutatjuk.
\end{abstract}

\section{Bevezetés, jelölések}

% jelolesek: Macbeath \cite{Ma67}

%FIXME Térkitöltésről általában

\section{D-szimbólum konstrukció adott térkitöltés alapján}
\cite{xy} alapján

Egy tetszőleges $d$-dimenziós geometriában adott kompakt fundamentális
tartományú térkitöltésből ($\mathcal{T}$) a következőképpen konstruáljuk meg a
D-szimbólumot:

\begin{enumerate}
  \item Bontsuk fel a fundamentális tartományt konvex részekre. (FIXME ez mindig
    megtehető?, Kell-e?)
  \item Vegyük a fundamentális tartomány konvex részeinek baricentrikus
    felbontásával kapott szimplexeket. A szimplexek csúcsai a következők szerint
    állnak elő: egy $d$-dimenziós konvex résztest belső pontja ($A_d$), ezen
    résztest egy $d-1$-dimenziós lapjának belső pontja ($A_{d-1}$),\ldots, ezen
    $2$-dimenziós lap egy $1$-dimenziós élének belső pontja ($A_1$) végül ezen
    élnek egy $0$-dimenziós csúcsa ($A_0$).
    Az így kapott szimpliciális felbontást jelöljük $\mathcal{C}$-vel; a
    szimplexet $C_j\in\mathcal{C}$-vel, ahol $j$ az adott szimplex sorszáma.
  \item Általában megkövetelhetjük, hogy a kövezésnek egy $\Gamma\leq
    \mathrm{Aut}\mathcal{T}$ szimmetriacsoportja változatlanul hagyja a kövezés
    kombinatorikus struktúráját, így a baricentrikus felbontást is.
    Ezzel az eredeti térkitöltésből egy szimpliciális térkitöltést kapunk, ezt
    szintén $\mathcal{C}$-vel jelöljük, illetve a szimpliciális térkitöltés
    szimplexeit $C_j\in\mathcal{C}$-vel.
  \item Bevezetünk szomszédsági operációkat az előbb előállított baricentrikus
    szimplexekre, $d$-dimenzióban:
    $\Sigma=\left\{\sigma_0,\sigma_1,\ldots,\sigma_d\right\}$. Az
    operációk jelentése:
    $\sigma_i(C)$ a $C\in\mathcal{C}$ szimplex azon szomszédja, amely az
    $i$-lapja, azaz az $A_0,\ldots,A_{i-1},A_{i+1},\ldots,A_d$ lap mentén
    szomszédos $C$-vel. Minden $\sigma_i$ operáció egy involúció a fent leírt
    szimplexek $\mathcal{C}$ halmazán. A szomszédsági operációk egyértelműen
    kiterjednek a szimpliciális térkitöltés szimplex-pályáira is.
    
    Az ábráinkon a következő konvenciót alkalmazzuk a legfeljebb 3 dimenziós
    szomszédsági operációk jelölésére:

    \setlength{\unitlength}{1cm}
    $\sigma_0$:
    \begin{picture}(1,0.2)
      \multiput(0,0.1)(0.2,0){5}{\circle*{0.001}}
    \end{picture},
    $\sigma_1$:
    \begin{picture}(1,0.2)
      \multiput(0,0.1)(0.25,0){4}{\line(1,0){0.15}}
    \end{picture},
    $\sigma_2$:
    \begin{picture}(1,0.2)
      \put(0,0.1){\line(1,0){1}}
    \end{picture},
    $\sigma_3$:
    \begin{picture}(1.5,0.2)
      \multiput(0,0.1)(0.5,0){3}{\line(1,0){0.2}}
      \multiput(0.35,0.1)(0.5,0){3}{\circle*{0.001}}
    \end{picture}

  \item 
\end{enumerate}

\subsection{Példa}
%négyzet alapú piramis (teljes szimmetriatöréssel) fund. tart.-u kocka kövezés

\subsection{Általános feltételek}
\subsubsection{Diagram}
\subsubsection{Mátrix-függvény}
\subsubsection{Alacsonyabb dimenziós komponensek}
\subsection{További megszorítások él- illetve laptranzitív esetben}

\section{D-szimbólumok algoritmikus felsorolása és részleges jellemzése}
\subsection{D-diagramok felsorolása}
Mikor nevezzünk 2 diagramot külöbözőnek? Bevezethető egy lexikografikus rendezés.
Összefüggő diagram "helyes" sorszámozása adott kezdőpont esetén.
Korábbi feltételek!
\subsection{Mátrix-függvények felsorolása}
%Alsó korlát: splitting
%Felső korlát: Alacsonyabb dimenziós komponensek
\subsection{További megszorítások él- illetve laptranzitív esetben}
%Bizonyítható végesség
\subsection{Jellemzés}
\subsection{Példa}
%Előző példa ügyesen visszafelé...

\section{Továbblépési lehetőségek}

%Bibiliography
\nocite{DHM93,D87,Du88,H93,LM90,Ma67,M94,T82,VS93,F94,F03}
\bibliographystyle{plain}
\bibliography{dsym}

\end{document}
