\documentclass[12pt,magyar,a4paper]{article}
\usepackage{graphicx}
\usepackage[magyar]{babel}
\usepackage[T1]{fontenc}
\usepackage[utf8]{inputenc}
\usepackage{verbatim}
\usepackage[colorlinks=true,linkcolor=black,urlcolor=blue]{hyperref}
\usepackage{listings}
\usepackage{amsmath}
\usepackage{amssymb}
\usepackage{ae,aecompl}

\hyphenpenalty=5000
\tolerance=1000

\begin{document}

\title{FIXME D-szimbólumok algoritmikus vizsgálata}
\author{Boróczki, Lajos}
\date{2011 február}
\maketitle

\begin{abstract}
\end{abstract}

\section{Bevezetés}

\section{FIXME Térkitöltésről általában}

\section{D-szimbólum konstrukció adott térkitöltés alapján}
\subsection{Általános feltételek}
\subsubsection{Diagram}
\subsubsection{Mátrix-függvény}
\subsubsection{Alacsonyabb dimenziós komponensek}
\subsection{További megszorítások él- illetve laptranzitív esetben}
\subsection{Példa}
%négyzet alapú piramis (teljes szimmetriatöréssel) fund. tart.-u kocka kövezés

\section{D-szimbólumok algoritmikus felsorolása és részleges jellemzése}
\subsection{D-diagramok felsorolása}
Mikor nevezzünk 2 diagramot külöbözőnek? Bevezethető egy lexikografikus rendezés.
Összefüggő diagram "helyes" sorszámozása adott kezdőpont esetén.
Korábbi feltételek!
\subsection{Mátrix-függvények felsorolása}
%Alsó korlát: splitting
%Felső korlát: Alacsonyabb dimenziós komponensek
\subsection{További megszorítások él- illetve laptranzitív esetben}
%Bizonyítható végesség
\subsection{Jellemzés}
\subsection{Példa}
%Előző példa ügyesen visszafelé...

\section{Továbblépési lehetőségek}

%Bibiliography
\nocite{DHM93,D87,Du88,H93,LM90,Ma67,M94,T82,VS93,F94,F03}
\bibliographystyle{plain}
\bibliography{dsym}

\end{document}
