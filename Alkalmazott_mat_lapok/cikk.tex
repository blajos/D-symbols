\documentclass[12pt,magyar,a4paper]{article}
\usepackage{graphicx}
\usepackage[magyar]{babel}
\usepackage[T1]{fontenc}
\usepackage[utf8]{inputenc}
\usepackage{verbatim}
\usepackage[colorlinks=true,linkcolor=black,urlcolor=blue]{hyperref}
\usepackage{listings}
\usepackage{amsmath}
\usepackage{amssymb}
\usepackage{ae,aecompl}

\newcommand{\leftexp}[2]{{\vphantom{#2}}^{#1}{#2}}
\newtheorem{thm}{Tétel}[section]
\newtheorem{defn}[thm]{Definíció}
\newtheorem{lem}[thm]{Lemma}
\newtheorem{kov}[thm]{Következmény}
\newtheorem{sej}[thm]{Sejtés}
\newtheorem{alg}[thm]{Algoritmus}
\newenvironment{biz}{Bizonyítás: }{Q.E.D.}


%\hyphenpenalty=5000
%\tolerance=1000

\begin{document}

\title{FIXME D-szimbólumok algoritmikus vizsgálata}
\author{Boróczki, Lajos}
\date{2011 február}
\maketitle

\begin{abstract}
  A térkitöltések kezelésének egy algebrai módszerével, a Delone-Delaney-Dress
  szimbólumokkal (röviden D-szimbólum) foglalkozunk. A módszer (\cite{xy}
  alapján) a térkitöltéseket egy több színű gráffal és a hozzá tartozó
  mátrix-függvényekkel írja le.

  Alapvető célunk a 3 dimenziós D-szimbólumok algoritmikus felsorolása, illetve
  a hozzájuk tartozó lehetséges kövezések minél részletesebb jellemzése.
  Végtelen sok D-szimbólum van rögzített dimenzióban is, ezért bemutatunk egy
  olyan megszorítási lehetőséget, ami garantálja a véges sok esetet, de
  továbbra is érdekes. Általános esetben jelenleg legfeljebb 12, éltranzitív
  esetben a 24-ből legfeljebb 18 elemű D-szimbólumokat tudjuk felsorolni.

  A D-szimbólumok által leírt kövezések pontos geometriai jellemzőit egyelőre
  nem tudjuk algoritmikusan leírni, ugyanis ez a Thurston-sejtés témaköréhez
  vezet, aminek nem ismert konstruktív bizonyítása (\cite{xy}). Ezért
  néhány esetben a kézzel történő jellemzést megmutatjuk.
\end{abstract}

\section{Bevezetés, jelölések}

% jelolesek: Macbeath \cite{Ma67}

%FIXME Térkitöltésről általában

\section{D-szimbólum konstrukció adott térkitöltés alapján}
\cite{xy} alapján

Egy tetszőleges $d$-dimenziós geometriában adott kompakt fundamentális
tartományú térkitöltésből ($\mathcal{T}$) a következőképpen konstruáljuk meg a
D-szimbólumot (FIXME a fundamentális tartománynak véges sok belső szimmetriája
lehet):

\begin{enumerate}
  \item Bontsuk fel a fundamentális tartományt konvex részekre. (FIXME ez mindig
    megtehető?, Kell-e?)
  \item Vegyük a fundamentális tartomány konvex részeinek baricentrikus
    felbontásával kapott szimplexeket. A szimplexek csúcsai a következők szerint
    állnak elő: egy $d$-dimenziós konvex résztest belső pontja ($A_d$), ezen
    résztest egy $d-1$-dimenziós lapjának belső pontja ($A_{d-1}$),\ldots, ezen
    $2$-dimenziós lap egy $1$-dimenziós élének belső pontja ($A_1$) végül ezen
    élnek egy $0$-dimenziós csúcsa ($A_0$).
    Az így kapott szimpliciális felbontást jelöljük $\mathcal{C}$-vel; a
    szimplexet $C_j\in\mathcal{C}$-vel, ahol $j$ az adott szimplex sorszáma.
  \item Általában megkövetelhetjük, hogy a kövezésnek egy $\Gamma\leq
    \mathrm{Aut}\mathcal{T}$ szimmetriacsoportja változatlanul hagyja a kövezés
    kombinatorikus struktúráját, így a baricentrikus felbontást is.
    Ezzel az eredeti térkitöltésből egy szimpliciális térkitöltést kapunk, ezt
    $\mathcal{C}^\Gamma$-val jelöljük, illetve a szimpliciális térkitöltés
    szimplexeit $C_j^\Gamma\in\mathcal{C}^\Gamma$-val.
  \item Bevezetünk szomszédsági operációkat az előbb előállított baricentrikus
    szimplexekre, $d$-dimenzióban ($I=\{0,1,\ldots,d\}$ indexhalmaz):
    $\Sigma_I=\left\{\sigma_0,\sigma_1,\ldots,\sigma_d\right\}$. Az
    operációk jelentése:
    $\sigma_i(C)$ a $C\in\mathcal{C}$ szimplex azon szomszédja, amely az
    $i$-lapja, azaz az $A_0,\ldots,A_{i-1},A_{i+1},\ldots,A_d$ lap mentén
    szomszédos $C$-vel. Minden $\sigma_i$ operáció egy involúció a fent leírt
    szimplexek $\mathcal{C}$ halmazán. A szomszédsági operációk egyértelműen
    kiterjednek a szimpliciális térkitöltés szimplex-pályáira is.
    
    Az ábráinkon a következő konvenciót alkalmazzuk a legfeljebb 3 dimenziós
    szomszédsági operációk jelölésére:

    \setlength{\unitlength}{1cm}
    $\sigma_0$:
    \begin{picture}(1,0.2)
      \multiput(0,0.1)(0.2,0){5}{\circle*{0.001}}
    \end{picture},
    $\sigma_1$:
    \begin{picture}(1,0.2)
      \multiput(0,0.1)(0.25,0){4}{\line(1,0){0.15}}
    \end{picture},
    $\sigma_2$:
    \begin{picture}(1,0.2)
      \put(0,0.1){\line(1,0){1}}
    \end{picture},
    $\sigma_3$:
    \begin{picture}(1.5,0.2)
      \multiput(0,0.1)(0.5,0){3}{\line(1,0){0.2}}
      \multiput(0.35,0.1)(0.5,0){3}{\circle*{0.001}}
    \end{picture}

  \item Vizsgáljuk meg, hogy egy $(d-2)$-lap körül ($3$ dimenzióban él körül)
    hány poliéder szerepel, illetve hány baricentrikus szimplex csatlakozik.
    Tegyük fel, hogy minden $(d-2)$-lap körül véges sok poliéder van.  Ezt az
    egyszerűbb kezelhetőség miatt érdemes megtenni, bár hiperbolikus esetekben
    előfordulhatna végtelen sok is. Ez a következőt jelenti:
    \begin{equation}
      (\sigma_j\sigma_i)\cdots(\sigma_j\sigma_i)(C)=
      (\sigma_j\sigma_i)^m(C) =C
    \end{equation}
    valamilyen $m$ természetes számra $m$-szer alkalmazva a $(\sigma_j\sigma_i)$
    operációt. A legkisebb ilyen $m$ számot jelöljük $m_{ij}(C)=m_{ji}(C)$-vel
    (az index szimmetria nyilvánvaló.)

    Definiáljuk az $M$ mátrix-függvényt a következőképpen:
    \begin{equation}
      M:\;\mathcal{C}\rightarrow N_{I\times I},\quad
      M(C)=\left[m_{ij}(C)\right]
    \end{equation}

    A térkitöltés szimmetriája nem változtatja meg a $(d-2)$-lap körüli
    poliéderek, így a baricentrikus szimplexek számát sem, tehát az $M$
    mátrix-függvény kiterjeszthető a szimmetria csoport szerinti
    szimplex-pályákra is:
    \begin{equation}
      M:\;\mathcal{C}^\Gamma\rightarrow N_{I\times I},\quad
      M(C^\Gamma)=\left[m_{ij}(C^\Gamma)\right]                                                          
    \end{equation}

  \item Vezessük be azt a mátrix-függvényt is, ami egy $(d-2)$-lap körül
    csatlakozó szimplex-pályák számát mutatja meg:
    \begin{equation}
      $r_{ij}(C^\Gamma)=r_{ji}(C^\Gamma)=\mathrm{min}\left\{r\left|
      \left((\sigma_j\sigma_i)\cdots(\sigma_j\sigma_i)(C^\Gamma)=\right)(\sigma_j\sigma_i)^r(C^\Gamma)=C^\Gamma
      \right.\right\}
    \end{equation}

    Definiáljuk az $R$ mátrix-függvényt a következőképpen:
    \begin{equation}
      R:\;\mathcal{C}^\Gamma\rightarrow N_{I\times I},\quad
      R(C^\Gamma)=\left[r_{ij}(C^\Gamma)\right]
    \end{equation}

    Azonnal látszik, hogy $\forall C^\Gamma\in\mathcal{C}^\Gamma,\,\forall
    i,j\in I$ esetén az alábbi oszthatóság teljesül:
    \begin{equation}
      \left.r_{ij}(C^\Gamma)\right|m_{ij}(C)
    \end{equation}
\end{enumerate}

Az így kapott $\mathcal{C}^\Gamma, \Sigma_I, M$ hármast nevezzük D-szimbólumnak. A
$\mathcal{C}^\Gamma, \Sigma_I$ párt egy $d+1$ színű diagrammal ábrázoljuk; ahol
minden csúcs (szimplex-pályák) minden szín szerinti foka 1 vagy 2, pontosan
akkor 2, ha az adott él egy hurok.

\begin{defn}
  Egy D-szimbólumot összefüggőnek nevezünk, ha a $\mathcal{C}^\Gamma, \Sigma_I$
  diagram, mint gráf összefüggő.
\end{defn}

\begin{defn}
  A $\mathcal{C}^\Gamma, \Sigma_I, M$ D-szimbólum duálisa az a D-szimbólum,
  amiben az $i$-edik és a $d-i$-edik szomszédsági operációt, illetve a
  megfelelő mátrix-függvény sorokat és oszlopokat felcseréljük. FIXME képlettel.
\end{defn}

\subsection{Példa}
%négyzet alapú piramis (teljes szimmetriatöréssel) fund. tart.-u kocka kövezés
Vegyük a \ref{pelda1} ábrán látható szimmetriatöréses kocka kövezést. Ennek
D-szimbóluma a \ref{pelda1_D-sym} ábrán látható.

\subsection{Általános feltételek}
A \ref{construct} részben leírt konstrukció és a tény, hogy ezzel egy
térkitöltést írunk le, a következő feltételeket indukálja egy tetszőleges
D-szimbólumra.

Vegyünk egy tetszőleges $\mathcal{D}, \Sigma_I, M$ hármast. Ahol $\mathcal{D}$
véges, $I$ a dimenzió szerinti index-halmaz ($I=\{0,1,\ldots,d\}$ $d$
dimenzióban),
$\Sigma_I=\left\{\sigma_i:\mathcal{D}\leftarrow\mathcal{D}\left|i\in
I\right.\right\}$ permutációk halmaza és $M: \mathcal{D} \leftarrow gl(d+1)$
mátrix-függvény $\mathcal{D}$ és $(d+1)\times(d+1)$-es mátrixok között.

A következő pontokban felsorolt feltételek teljesülése esetén nevezzük
D-szimbólumnak a fenti hármast:
\subsubsection{Diagram}
A $\mathcal{D}, \Sigma_I$ D-diagramra vonatkozó feltételek:
\begin{enumerate}
  \item A szimplex pályákon értelmezett szomszédsági operációk involutív
    permutációk, vagyis $\forall i\in I, \sigma_i\in \Sigma_I, \forall D\in
    \mathcal{D}$: $\sigma_i\sigma_i(D)=D$. A diagram minden csúcsában az azonos
    színű élek foka 1 vagy 2. A fok pontosan akkor 2, ha az él egy hurok (kezdő
    és végpontja azonos.)
  \item A baricentrikus felbontással való kompatibilitás miatt, a nem szomszédos
    operáció-párokat alkalmazva legfeljebb $2$ lépésből vissza kell jutnunk a
    kiindulási csúcsba. $\forall i,j\in I, \forall D\in \mathcal{D}$: $|i-j|\geq
    2 \Rightarrow (\sigma_j\sigma_i)^2(D)=D$
\end{enumerate}

\subsubsection{Mátrix-függvény}
A \ref{construct} részben leírt konstrukció szerinti $R$ mátrix-függvény
felírható a D-diagram alapján. $\forall i,j\in I, \forall D\in \mathcal{D}$:
$r_{ij}(D)=r_{ji}(D)=\mathrm{min}\left\{r\in
\mathbb{N}^+|(\sigma_j\sigma_i)^r(D)=D\right\}$

A D-diagram feltételei az $R$ mátrix-függvényre vetítve: $r_{ii}(D)=1$,
illetve $|i-j|\geq 2 \Rightarrow r_{ij}(D)\leq2$.

Feltételek a diagramhoz tartozó $M$ mátrix-függvényre:
\begin{enumerate}
  \item Az $R$ mátrix-függvény minden elemének osztania kell az $M$
    mátrix-függvény megfelelő elemét. Csak így biztosítható, hogy egy $d-2$
    dimenziós lap ($3$ dimenzióban él) teljes körüljárásakor azonos szimplex
    pályába érkezünk; ami pedig szükséges ahhoz, hogy azonos szimplexbe
    érkezhessünk. Az $M$ és $R$ mátrix-függvények értékeinek hányadosa a kövezés
    periodicitását mutatja az adott él körül. $\forall i,j\in I, \forall D\in
    \mathcal{D}$: $r_{ij}(D)|m_{ij}(D)$.
  \item Pálya feltétel: Minden operáció-párhoz és kiindulási diagram-csúcshoz
    definiálhatjuk a hozzá tartozó pályát: az összes diagram-csúcs, amit
    érintünk miközben az operációkat elvégezzük. Az $M$ mátrix-függvény
    értékeinek egy-egy pályán belül meg kell egyeznie, különben a kapott
    kövezésben függne az él körüli szimplexek száma a kezdő szimplextől. A
    pályát visszafelé bejárva is ugyanazt az utat kell megtennünk, ezért az $M$
    mátrix-függvény mátrixai szimmetrikusak.
    \begin{align*}                                                                             
      &\forall i,j\in I, \forall D\in \mathcal{D} \\
      &\mathcal{D}'=\left\{(\sigma_j\sigma_i)^k(D)|k\in
      \mathbb{N}\right\}\cup\left\{\sigma_i(\sigma_j\sigma_i)^k(D)|k\in
      \mathbb{N}\right\}\\
      &\forall D_1,D_2 \in \mathcal{D}'\\
      &m_{ij}(D_1)=m_{ij}(D_2)=m_{ji}(D_1)=m_{ji}(D_2)
    \end{align*}
  \item Az $M$ mátrix-függvény minden mátrixának főátlójában mindenhol $1$-esek
    állnak, ez biztosítja, hogy a szomszédsági operációk involúciók a szimplexek
    halmazán is.
  \item A főátlótól $1$ távolságra lévő helyeken a mátrix-függvény érték legyen
    legalább 2. Egyenlőség esetén degenerált, illetve mesterkélt esetekhez
    jutunk, melyekben például digonok is lehetnek lapok. (Szép kövezés esetén
    egy 3 dimenziós poliéder bármely csúcsában legalább 3 él és 3 lap
    találkozik, minden lapnak legalább 3 oldaléle van, minden élnél legalább 3
    test és 3 lap fut össze.)
    $\forall i\in I-\{0\}, \forall D\in \mathcal{D} m_{i(i-1)}(D)\geq 2$
  \item Az $M$ mátrix-függvény minden mátrixának főátlójától legalább 2
    távolságra lévő helyeken $2$-esek állnak, különben nem kapjuk vissza a
    baricentrikus felbontás kombinatorikus merőlegességeit.
    $\forall i,j\in I, \forall D\in \mathcal{D} m_{ij}(D)=2$
\end{enumerate}

\subsubsection{Alacsonyabb dimenziós komponensek vizsgálata}
Az eddigi megállapítások szükséges feltételek olyan $d$ dimenziós (nem
feltétlenül euklideszi) térkitöltés létezésére (amelynek D-szimbóluma
$\mathcal{D}, \Sigma_I, M$.) $3$ dimenziós esetben további követelményeket
tudunk megfogalmazni.

\begin{defn}
  Rész D-szimbólum: Az $i$-edik involúció elhagyásával keletkező diagramok
  halmaza $K$. Az $i$-edik komponens vagy rész D-szimbólumok:
  $\left\{\leftexp{c}(\Sigma_I^i,\mathcal{D}^i,M^i)|c \in K\right\}$. Ahol
  $\Sigma_I^i=\Sigma_I-{\sigma_i}$, $\mathcal{D}^i$ a $c$-edik komponensbeli
  csúcsok halmaza, és $M^i$ a komponens csúcsain értelmezett mátrix-függvény,
  amit $M$-ből a komponens csúcsaihoz tartozó mátrixok $i$-edik sorának és
  oszlopának törlésével kapunk.
\end{defn}

A rész D-szimbólum vizuális jelentése egy az adott $i$-indexű csúcsosztálybeli
csúcs körüli eggyel kisebb dimenziós felületen létrehozott ($3$-dimenzióban
$2$-dimenziós) kövezés D-szimbóluma a csúcs stabilizátora szerint. Ezért
$3$-dimenzióban egy $i$-indexű valódi szimplex-csúcs körüli rész D-szimbólum egy
szférikus kövezéshez kell tartozzon, egy ideális szimplex-csúcs körül euklideszi
kövezés alakul ki (pl. a hiperbolikus tér paraszféra (horoszféra) felületén;)
végül a hiperbolikus síkkövezéseket kizárjuk, mert modellen kívüli (végtelennél
távolabbi) pontot, mint $i$-csúcsot jellemezne.

A rész D-szimbólum magasabb dimenzióban is értelmezhető megszorításokat
jelentene, de mivel a térkitöltések teljes osztályozása csak $2$-dimenzióban
ismert, ezért jelenleg csak a $3$-dimenziós D-szimbólumok esetén tudjuk a
módszert jól használni.

További megszorításaink alapja a kombinatorikus görbületi függvény számolása
(\cite{Emil}). A rész D-szimbólum komponenseihez tartozó kövezésekben a
D-szimbólum alapján felírható a kombinatorikus görbületi függvény:
\begin{align*}
  K(\leftexp{c}{\mathcal{D}}^i)=\sum_{D\in
  \leftexp{c}{\mathcal{D}}^i}\left(-1+\sum_{\substack{0\le j<k\le d \\
  j,k\ne i}}\frac{1}{m_{jk}(D)}\right)
  \begin{array}{cccc}
    > & & S^2 \\
    = & 0 & \mathbb{E}^2 \\
    < & & H^2
  \end{array}
\end{align*}
Ez alapján eldönthető, hogy a rész D-szimbólum a 3 lehetséges $2$-dimenziós
felület közül melyiken valósulhat meg.

Szférikus síkon történő kövezésnek további feltétele, hogy az úgy nevezett
rossz orbifoldokat kizárjuk, azaz a következő lehetőségek hibásak (Convay
illetve Macbeath-féle jelöléseik alapján \cite{CM}):
\begin{align*}
  u=(+,0;[u];\{\}), & & 1<u;\\
  *u=(+,0;[];\{(u)\}), & & 1<u;\\
  uv=(+,0;[u,v];\{\}), & & 1<u<v;\\
  *uv=(+,0;[];\{(u,v)\}), & & 1<u<v.
\end{align*}
Ezek a csepp felületekre illetve az észak-déli póluson nem azonosan viselkedő
gömb felületekre utalnak.

További követelmények $3$-dimenziós D-szimbólumokra a rész D-szimbólumok alapján:
\begin{enumerate}                                                                                    
  \item $\leftexp{c}(\Sigma_I^1,\mathcal{D}^1,\mathcal{M}^1)$ és
    $\leftexp{c}(\Sigma_I^2,\mathcal{D}^2,\mathcal{M}^2)$ esetén a görbületi
    függvény pozitív és a jó orbifold feltétel teljesül minden komponensben.
    Élközépponthoz és lapközépponthoz tartozó szimplex csúcs valódi csúcs kell
    legyen, ezért egy $S^2$-n megvalósuló kövezést kell kapjunk.
  \item $\leftexp{c}(\Sigma_I^0,\mathcal{D}^0,\mathcal{M}^0)$ és
    $\leftexp{c}(\Sigma_I^3,\mathcal{D}^3,\mathcal{M}^3)$ esetén a görbületi függvény
    pozitív és a jó orbifold feltétel teljesül, vagy a görbületi függvény
    $0$ minden komponensben. Csúcshoz és test-középponthoz tartozó szimplex csúcs
    lehet valódi, ekkor egy $S^2$-n megvalósuló kövezést kell kapjunk; illetve
    eyekben az esetekben megengedjük hogy ideális pontok legyenek, ekkor
    $\mathbb{E}^2$-n megvalósuló kövezést kell kapjunk. Az ideális
    test-középpont esetét a dualitás miatt meghagyjuk; de megjegyezzük, hogy a
    végtelen poliéderekkel történő kövezés nem igazán érdekes (egy kisebb
    dimenziós felület kövezésének felel meg.)
\end{enumerate}

\subsection{További megszorítások él- illetve laptranzitív esetben}
Módszerünkkel könnyen vizsgálhatók a $3$ dimenziós él- illetve laptranzitív
kövezések egy egyszerű további feltétel bevezetésével. Az él- illetve
laptranzitív kövezések esetén egyetlen él- illetve lap-osztályról beszélünk, ami
D-szimbólumos terminológiában annyit jelent, hogy az $1$-es illetve a $2$-es
involúciót elhagyva a D-diagram továbbra is összefüggő marad.

Ezen esetek azért érdekesek, mert (\cite{aa} alapján) ismert, hogy legfeljebb
$|\mathcal{D}|=24$ elemű D-szimbólumok fordulhatnak elő laptranzitív esetben,
aminek duálisai az éltranzitív esetek, ezért ezesetben is legfeljebb
$|\mathcal{D}|=24$ elemű D-szimbólumokról van szó. Az összes lehetséges ilyen
D-szimbólum felsorolása a mostani számítástechnikai eszközökkel elérhető
közelségben van.

\section{D-szimbólumok algoritmikus felsorolása és részleges jellemzése}
\subsection{D-diagramok felsorolása}
Mikor nevezzünk 2 diagramot külöbözőnek? Bevezethető egy lexikografikus rendezés.
Összefüggő diagram "helyes" sorszámozása adott kezdőpont esetén.
Korábbi feltételek!
\subsection{Mátrix-függvények felsorolása}
%Alsó korlát: splitting
%Felső korlát: Alacsonyabb dimenziós komponensek
\subsection{További megszorítások él- illetve laptranzitív esetben}
%Bizonyítható végesség
\subsection{Jellemzés}
\subsection{Példa}
%Előző példa ügyesen visszafelé...

\section{Továbblépési lehetőségek}

%Bibiliography
\nocite{DHM93,D87,Du88,H93,LM90,Ma67,M94,T82,VS93,F94,F03}
\bibliographystyle{plain}
\bibliography{dsym}

\end{document}
