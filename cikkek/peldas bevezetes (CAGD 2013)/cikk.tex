\documentclass[12pt,a4paper]{article}
\usepackage{graphicx}
\usepackage[english]{babel}
\usepackage[T1]{fontenc}
\usepackage[utf8]{inputenc}
\usepackage{verbatim}
\usepackage[colorlinks=true,linkcolor=black,urlcolor=blue]{hyperref}
\usepackage{bookmark}
\usepackage{listings}
\usepackage{amsmath}
\usepackage{amssymb}
\usepackage{ae,aecompl}

\newcommand{\leftexp}[2]{{\vphantom{#2}}^{#1}{#2}}
\newtheorem{thm}{Theorem}[section]
\newtheorem{defn}[thm]{Definition}
\newtheorem{lem}[thm]{Lemma}
\newtheorem{kov}[thm]{Corollary}
\newtheorem{sej}[thm]{Conjecture}
\newtheorem{alg}[thm]{Algorithm}
\newenvironment{biz}{Proof: }{\qed}


\begin{document}

\title{Application of D-symbols on edge transitive orbifolds}
\author{Boróczki, Lajos}
\maketitle
%\tableofcontents

\begin{abstract}
  Any tiling, generated by a crystallographic group with compact fundamental
  domain, can be represented by a diagram and a matrix valued function, based on
  their barycentric subdivision and the adjacency relations between the orbits
  and the particular simplices. The representation is called the D-symbol of a
  tiling in honour of Delone, Delaney and Dress. The representation is easily
  adoptable to computer programs.

  After a short introduction and definition of D-symbols we would like to
  present some recent examples of edge transitive D-symbols and some of their
  nice properties found by computer analysis.
\end{abstract}

\section{Introduction}
This work is based on...

Why D-symbol?

Unpublished accomplishments?

Why just some examples?

\section{D-symbols}
We shall briefly introduce D-symbols (for a more comprehensive introduction see
also \cite{BSzK02}, \cite{DHM93}, \cite{M96}, FIXME). 

For an equivariant periodic tiling $(\mathcal{T},\Gamma)$ on a $d$-dimensional simply
connected manifold $\mathcal{S}^d$ the barycentric simplex subdivision of $\mathcal{T}$
can be taken, which remains compatible with the action of $\Gamma$. Let's denote
the resulting chamber system with $\mathcal{C}$, the $\Gamma$ orbits of the
chambers with $C^\Gamma$ for every $C\in \mathcal{C}$ and $\mathcal{D}:=
\mathcal{C}^\Gamma$. Every $C\in \mathcal{C}$
is a $d$-dimensional simplex, and the vertices of the simplex can be labeled by
the numbers $0\ldots d$ showing the dimension of their respective face. For
example if we take the baricentric subdivision of a $3$-dimensional cube: we get
a body center labeled by $3$, a facet center labeled by $2$, an edge center
labeled by $1$ and one of the original vertices labeled by $0$. The facets
of the chambers ($d-1$ dimensional faces) can be labeled by the same $0\ldots d$
using the label of the only vertex, that the facet doesn't contain.

Every chamber has $d+1$ neighbours by the facet they share and this adjacency is
conserved by the action of $\Gamma$. So we can take the adjacency operations and
use them formally as a free Coxeter group $\Sigma^I := <(\sigma^i, i\in I) |
((\sigma^i)^2=1, i\in I)>$ acting on $\mathcal{C}$ or $\mathcal{D}$
respectively, where $I$ is the index set of $0\ldots d$. Note 1: It's possible to
get back to the starting chamber (or starting chamber-orbit) without $\sigma$
being $1$ for example take a route around an edge of the starting chamber. Note
2: As we only consider periodic tilings, $\mathcal{D}$ is finite, this
automatically defines some relations of $\Sigma^I$ affecting $\mathcal{D}$.

Next we have to deal with the remaining relations. We require that the following
relations hold for every $C\in\mathcal{C}$:
\begin{itemize}
  \item For every $i,j\in I$, where $|i-j|=1$, there exists
    $m\in\mathbb{N}\setimnus {0}$:
    $(\sigma^j\sigma^i)^mC=(\sigma^i\sigma^j)^mC=C$, lets denote the smallest
    possible $m$ value with $m_{ij}(C)=m_{ji}(C)$.
  \item For every $i,j\in I$, where $|i-j|>1$:
    $(\sigma^j\sigma^i)^2C=(\sigma^i\sigma^j)^2C=C$, $m_{ij}(C)=m_{ji}(C)=2$, for the compatibility with
    baricentric subdivision.  
  \item (For every $i\in I$: $(\sigma^i)^2C=C$, $m_{ii}(C)=1$, which is already defined.)
\end{itemize}

We can introduce the matrix valued function $\mathcal{M}: \mathcal{C}
\rightarrow \mathbb{N}^{d\times d}$: $\mathcal{M}(C)_{i,j}=m_{ij}(C)$, which
gives us half the number of chambers around an edge. Notice that $\mathcal{M}$
is the same for the orbits ($\mathcal{D}$), because the action of $\Gamma$
doesn't change the number of chambers around a given edge.

Similarly one can define the $\mathcal{R}$ matrix valued function for the
relations of the $\Gamma$ orbits of the chambers ($\mathcal{D}$), which gives us
half the number of chamber-orbits around an edge. Every value of the $\mathcal{R}$
matrix valued function has to divide the respective value of $\mathcal{M}$ for
every $D\in\mathcal{D}$, for every $i,j\in I$ exists the value $r$ such that
$(\sigma^j\sigma^i)^rD=(\sigma^i\sigma^j)^rD=D$. Lets denote the smallest
possible $r$ value with $r_{ij}(D)=r_{ji}(D)$. There exists $C\in\mathcal{C}$ so
that $D=C^\Gamma$.
\begin{itemize}
  \item If $|i-j|=1$: $r_{ij}(D)|m_{ij}(C)$
  \item If $|i-j|>1$: $r_{ij}(D)=1$ or $r_{ij}(D)=2$
  \item If $i=j$: $r_{ij}(D)=1$
\end{itemize}

It's easy to see that the values of both matrix valued functions must be the
same around an edge ($D,D'\in\mathcal{D}$, $C,C'\in\mathcal{C}$, $i,j\in I$,
$a\in {0,1}$, $b\in \mathbb{N}$):
\begin{eqnarray}
  D'=(\sigma^i)^a(\sigma^i\sigma^j)^b(D) & \Rightarrow &
  \mathcal{R}(D)_{i,j}=\mathcal{R}(D')_{i,j} \\
  D'=(\sigma^i)^a(\sigma^i\sigma^j)^b(D) & \Rightarrow &
  \mathcal{M}(D)_{i,j}=\mathcal{M}(D')_{i,j} \\
  C'=(\sigma^i)^a(\sigma^i\sigma^j)^b(C) & \Rightarrow &
  \mathcal{M}(C)_{i,j}=\mathcal{M}(C')_{i,j} \\
\end{eqnarray}

If we take a small ball around every chamber-vertex (so small, that it doesn't
contain another chamber vertex), then the tiling on the boundary of the ball
induced by the original tiling must be an $S^{d-1}$ tiling \cite{D87} if the
chamber vertice is a proper vertice (not an ideal or an out-of-model vertice).

The tuple $(\Sigma^I,\mathcal{D},\mathcal{M})$ is called the D-symbol of a
tiling in honour of Delone, Delaney and Dress. (FIXME: $\mathcal{M}$-et bele
lehetne olvasztani a $\Sigma^I$-be relaciokkent.) $(\mathcal{T},\Gamma)$ and
$(\mathcal{T}',\Gamma')$ are equivalent (topologically equivariant (homeomeric))
if and only if the correspondig $(\Sigma^I,\mathcal{D},\mathcal{M})$ and
$(\Sigma^I',\mathcal{D}',\mathcal{M}')$ D-symbols are isomorphic \cite{D87}.

\section{Inverse problem}
Our main concern is the inverse problem: Which tuples
$(\Sigma^I,\mathcal{D},\mathcal{M})$ exist with some nice constraints and how
can we find the orbifold and the geometry it realizes?

Let's recall the definitions and theorems used in this paper.
\begin{defn}
  A $d$-dimensional \em{D-diagram} is a $d+1$ colored graph of the adjacency
  operations $(\Sigma^I,\mathcal{D})$. Where $|I|=d+1$, $|\mathcal{D}|$ is the
  finite \em{cardinality} of the diagram, for every $i\in I$ $\sigma^i^2=1$ and
  for every $i,j\in I$ if $|i-j|>1$ then $(\sigma^i\sigma^j)^2D=D$. We use the
  following colors on our figures up to 3 dimensions:

  \setlength{\unitlength}{1cm}
  $\sigma^0$:
  \begin{picture}(1,0.2)
    \multiput(0,0.1)(0.2,0){5}{\circle*{0.001}}
  \end{picture},
  $\sigma^1$:
  \begin{picture}(1,0.2)
    \multiput(0,0.1)(0.25,0){4}{\line(1,0){0.15}}
  \end{picture},
  $\sigma^2$:
  \begin{picture}(1,0.2)
    \put(0,0.1){\line(1,0){1}}
  \end{picture},
  $\sigma^3$:
  \begin{picture}(1.5,0.2)
    \multiput(0,0.1)(0.5,0){3}{\line(1,0){0.2}}
    \multiput(0.35,0.1)(0.5,0){3}{\circle*{0.001}}
  \end{picture}
\end{defn}

\begin{defn}
  \em{$\mathcal{R}$ matrix valued function.}
  Let $\mathcal{R}$: $\mathcal{D} \rightarrow \mathbb{N}^{d\times d}$.
  For every $D\in\mathcal{D}$, for every $i,j\in I$ there exists $r\in
  \mathbb{N}\setminus{0}$ such that
  $(\sigma^j\sigma^i)^rD=(\sigma^i\sigma^j)^rD=D$ ($|\mathcal{D}|<\infinity$).
  $\mathcal{R}(D)_{i,j}:=r$
\end{defn}
  
\begin{defn}
  \em{$\mathcal{M}$ matrix valued function.}
  Let $\mathcal{M}$: $\mathcal{D} \rightarrow \mathbb{N}^{d\times d}$.
  For every $D\in\mathcal{D}$,
  \begin{itemize}
    \item for every $i\in I$: $\mathcal{M}(D)_{i,i}=1$
    \item for every $i,j\in I$: if $|i-j|>1$ then $\mathcal{M}(D)_{i,j}=2$
    \item for every $i,j\in I$: if $|i-j|=1$ then $\mathcal{R}(D)_{i,j}|\mathcal{M}(D)_{i,j}$
  \end{itemize}
\end{defn}

\begin{defn}
  \em{$\mathcal{P}$ matrix valued function of parameters.}
  Let $\mathcal{P}$: $\mathcal{D} \rightarrow \mathbb{N}^{d\times d}$.
  For every $D\in\mathcal{D}$, for every $i,j\in I$
  $\mathcal{P}(D)_{i,j}=\frac{\mathcal{M}(D)_{i,j}}{\mathcal{R}(D)_{i,j}}$.
\end{defn}

\begin{defn}
  We call the tuple $(\Sigma^I,\mathcal{D},\mathcal{M})$ \em{D-symbol}, if the
  following additional constraints are met:
  \begin{itemize}
    \item $\forall D\in \mathcal{D}$, $\forall i,j\in I$, $\forall a\in {0,1}$,
      $\forall b\in \mathbb{N}$: if $D'=(\sigma^i)^a(\sigma^i\sigma^j)^b(D)$,
      then $\mathcal{M}(D)_{i,j}=\mathcal{M}(D')_{i,j}$, namely the
      $\mathcal{M}$ matrix function has the same values around edges.
    \item FIXME(talan nem felejtettem ki semmi olyat, amit nem akartam)
  \end{itemize}
\end{defn}

\begin{defn}
  Parametric D-symbol
\end{defn}

\begin{defn}
  D-subsymbol
\end{defn}
Note 1: tiling around vertex d-1 ball

Note 2: Number of components: number of chamber-vertex classes

curvature function

\begin{defn}
  Proper 3 dimensional D-symbol: S2 (bad orbifold problem), E2
\end{defn}
Note: Higher dimensions...

From now on we only consider proper $3$-dimensional D-symbols.

\begin{defn}
  homomorphism(isomorphism)
\end{defn}

\begin{defn}
  maximality
\end{defn}

\begin{defn}
  duality
\end{defn}

\subsection{Edge transitive D-symbols}
Extra Constraint

Finite

\section{Examples}
d3c24\_2 (d3c12\_30), d3c24\_30 (d3c12\_61)

Start with cardinality 24 as there are maximal examples, but go down to
cardinality 6 or 3 so it's easier to see... Graphics

R matrix functions. Some possible M matrix functions (maximal, non max): Let's
see a sphere's curvature function around a vertex (smaller cardinality?).

Bad orbifolds?

Infinite parameters WTF?

Ideal vertices?


\section{Future work}
Analyse every edge-transitive symbol

Check 2 dimensional tilings on the border of partitions (splitting/fiber)

Difficult part: find out the signature and type of underlying geometries

%Bibiliography
\nocite{DHM93,D87,Du88,H93,LM90,Ma67,M94,T82,VS93,F94,F03}
\bibliographystyle{plain}
\bibliography{dsym}

\end{document}
