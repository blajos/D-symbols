\documentclass[a4paper]{article}
\usepackage[english]{babel}
\usepackage[T1]{fontenc}
\usepackage[utf8]{inputenc}
\begin{document}
\small
Conference on Geometry: Theory and Applications\\
CGTA 2013, Ljubjana/Slovenia, June 24-28, 2013
\vfill
\large
\begin{center} 
  On the splitting problem of orbifolds via D-symbols
\end{center}
\vfill
\small
\textbf{BORÓCZKI, Lajos}\\
Budapest University of Technology and Economics, MI Department of Geometry\\
E-mail: boroczki@math.bme.hu\\
http://www.math.bme.hu/\textasciitilde geometry\\
http://www.math.bme.hu/\textasciitilde boroczki\\
Project leader: dr. Molnár, Emil
\vfill
\small
\textbf{Abstract.} Any tiling, generated by a crystallographic group with
compact fundamental domain, can be represented by a diagram and a matrix valued
function, based on their barycentric subdivision and the adjacency relations
between the orbits and the particular simplices. The representation is called
the D-symbol of a tiling in honour of Delone, Delaney and Dress. The
representation is easily adoptable to computer programs.

Based on Thurston's geometrization conjecture there are 8 possible geometric
structures on special 3-manifolds which are cut along tori. There exists at
least 4 proof of the theorem but none of them is constructive. Based on our
method it would be easier to find a tiling which does not fit in any of the 8
geometries; but inspecting the tilings we can possibly move forward to a
constructive proof of the theorem.

Using D-symbols one can examine the properties of orbifolds, but luckily the
3-manifolds in the conjecture are trivially orbifolds. There are infinitely many
D-symbols, but they can be enumerated based on their cardinality. So it may be
possible to enumerate every 3-manifold and verify the conjecture.

But first we have to find the "cuts along tori" which are called splittings in
our theory. We would like to present an algorithm with some examples for finding
every possible splittings of a D-symbol by examining the signature of
2-dimensional subtilings.

The future: After splitting D-symbols along the previously found splittings we
have to be able to tell the signature of the underlying projective space of the
primitive D-symbols.
\vfill
\vfill
\end{document}
