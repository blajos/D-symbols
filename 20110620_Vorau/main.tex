\usepackage[english]{babel}
\usepackage[T1]{fontenc}
\usepackage[utf8]{inputenc}
\usepackage{graphicx}
\usepackage{listings}
\usepackage{amsmath}
\usepackage{amssymb}
%\usepackage{amsthm}
\usepackage{ae,aecompl}

\usetheme{Goettingen}
%\usetheme{Singapore}

\title{Algorithms for finding 3 dimensional D-symbols}
\author{Boróczki, Lajos}
\date{2011-06-20}

\begin{document}

\begin{frame}
  \maketitle
\end{frame}

\begin{frame}
  \mode<presentation>{\frametitle{Outline}}
  \tableofcontents
\end{frame}
\newpage

\section{D-symbols}
\begin{frame}
  D-symbols:
  \begin{itemize}
    \item Based on the baricentric subdivision of a tiling
    \item Structure:
      \begin{itemize}
	\item D-diagram: $dim+1$ colored graph, which represents adjacencies of
	  simplex-orbits
	\item Matrix function on simplex orbits, which represents the number of
	  simplexes (not orbits) around a $dim-2$ dimensional edge.
      \end{itemize}
    \item Conditions:
      \begin{itemize}
	\item Compatibility between the diagram and the matrix function.
	\item Compatibility with baricentric subdivision
      \end{itemize}
  \end{itemize}
\end{frame}

\section{Finding D-diagrams}
\begin{frame}
  Simple algorithm for finding D-diagrams:
  \begin{itemize}
    \item Fixed dimension and cardinality
    \item Take $car$ number of graph vertices and $dim+1$ number of colors
    \item Enumerate every possible edge-combinations and on the way drop the
      "bad" and duplicate combinations. (A little perturbation for
      edge-transitive diagrams is possible.)
    \item Conditions:
      \begin{itemize}
	\item Rank of vertices
	\item Patterns which makes the diagram incompatible with baricentric
	  subdivision
	\item Is connected?
	\item Permutation of vertices (ordering of diagrams; and always take the
	  smallest empty vertex)
      \end{itemize}
  \end{itemize}
\end{frame}

\section{Finding matrix-functions}
\begin{frame}
  Algorithm for finding possible matrix-functions of a D-diagram in
  $3$-dimensions:
  \begin{itemize}
    \item Parametric matrix function, minimal values
    \item $2$-dimensional components (simplex point stabilizers):
      \begin{itemize}
	\item Based on combinatorial curvature (decreases if we increase the
	  parameter values)
	\item Only spherical or euclidean
	\item If spherical we have to exclude bad orbifolds
      \end{itemize}
    \item Increase every parameter one by one, manage infinite series
  \end{itemize}
  Some easy to answer questions:
  \begin{itemize}
    \item Does the D-symbol have inner symmetries (diagram and matrix function)?
    \item Does the D-symbol have ideal simplex points?
  \end{itemize}
\end{frame}

\end{document}
